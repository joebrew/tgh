\documentclass{article}

\usepackage{style}


\usepackage{Sweave}
\begin{document}

\Sconcordance{concordance:can_we_do_it.tex:can_we_do_it.Rnw:%
1 5 1 1 0 286 1}


\vspace{20mm}


\begin{Large}
\begin{center}
Literature review and research protocol
\end{center}
\end{Large}


\begin{large}
\begin{center}
\textbf{Can we do it? A systematic and reproducible survey of research professionals on the perceived feasibility of and obstacles to eliminating neglected tropical diseases and malaria using a “wisdom of crowds” approach} 
\end{center}
\end{large}


\vspace{5mm}

\begin{changemargin}{2.5cm}{2.5cm} 
\begin{center}
\begin{large}
Celine Aerts \hfill \emph{celine.aerts@isglobal.org } \\ 
Joe Brew \hfill \emph{joe.brew@isglobal.org} \\
Elisa Sicuri \hfill \emph{elisa.sicuri@isglobal.org} \\ 
\end{large}
\end{center}
\end{changemargin}


\vspace{6mm}

\begin{center}
\begin{large}
Institut de Salut Global de Barcelona 
\end{large}
\end{center}


\begin{changemargin}{3cm}{3cm} 

\begin{center}
\textbf{Summary}
\end{center}

\emph{In recent years, much of the discourse regarding neglected tropical diseases has shifted from "control" to "eradication." The emphasis on elimination serves to rally funder support, motivate researchers, and focus the efforts of public health practitioners. Proponents of disease eradication point to the success of historical and current campaigns (smallpox and polio, respectively), and highlight the benefits in health and wealth to future generations. However, the opportunity cost of investments in eradication-specific interventions is high, and the expected value of these interventions is a function of their likelihood of success. In a systematic survey of experts in the field of neglected tropical diseases (NTDs), we query beliefs regarding the likelihood and time-frame of eradication, as well as the perceived chief obstacles faced by those striving to eradicate NTDs. We assess pessimism/optimism (via the proxy of years-to-eradication), broken down by academic discipline, researcher impact, and years of experience. Our results serve as a barometer of professional opinion, and identify areas of research where experts in the field expect the most resistance.}
\end{changemargin}
\vfill  

\newpage


\section*{Executive summary}

This document is divided into two sections:

\begin{enumerate}
\item A literature review, which provides a justification for the proposal and situates this research within the relevant literature
\item A proposed protocol for review by and approval of the ISGlobal scientific committee
\end{enumerate}



\vspace{5mm}

\tableofcontents


\newpage  
\section*{Literature review}
\addcontentsline{toc}{section}{Literature Review}


\subsection*{Background}
\addcontentsline{toc}{subsection}{Background}

In recent years, researchers, public health agencies and funding organizations have become increasingly interested in transitioning from an approach of malaria "control" to one of "elimination" and "eradication".\cite{Tanner2015} Even in areas of high endemicity, advances in immunology, parasitology, modeling and vaccinology, along with rapid economic development, have made eradication appear a more feasible goal. \cite{Snow2015, Eckhoff2014}  \\

\noindent The economic case for striving to achieve malaria eradication is compelling \cite{Barofsky2015} \\

\noindent Most of the current research on expert opinion regarding the feasibility of malaria eradication focuses on the \emph{how} rather than the \emph{if}.\cite{Tanner2015} 

\subsection*{Methods}
\addcontentsline{toc}{subsection}{Methods}

\subsection*{Results}
\addcontentsline{toc}{subsection}{Results}

\subsection*{Discussion}
\addcontentsline{toc}{subsection}{Discussion}


\newpage
\section*{Protocol}
\addcontentsline{toc}{section}{Protocol}


\subsection*{Background}
\addcontentsline{toc}{Background}{}

\noindent \textbf{The wisdom of crowds:} Patients often ask for a “second opinion”, a request which implicitly recognizes two important truths: (1) that an expert can sometimes be wrong and (2) that the combined opinions of multiple experts can better approximate the truth than the opinion of only one. As Sir Francis Galton demonstrated in his famous ox-weight experiment published in Nature1, averaging the opinions of many is more accurate than taking the opinion of any single expert, since the biases of diverse viewpoints can be complementary and symbiotic. \\

\noindent \textbf{The value of forecasting:} In regards to disease eradication, proponents point to the potential ongoing returns on investment to future generations. But economically, the “expected value” of an investment in a binary scenario (eradication or not) is a function of the probability of the scenario’s occurrence, and the temporal lag of that occurrence. Therefore, knowing the likelihood and time-frame of eradication of NTDs and malaria is essential for making sound investments in health. \\

\noindent \textbf{Why this study:} Assessing likelihood and time-frame of eradication is too important of a task to be left to individuals or small panels and committees. It requires the “wisdom of crowds.” Measuring consensus and discord among disease-specific researchers from a variety of disciplines can serve as a barometer of (informed) opinion, both guiding resources and identifying areas of concern.

\subsection*{Objectives}
\addcontentsline{toc}{subsection}{Objectives}


In a systematic survey of experts in the fields of neglected tropical diseases and malaria, we will query perceptions regarding the feasibility and time-frame of eradication, as well as the perceived gaps and chief areas that need attention in order for eradication to occur. We will report on aggregate results, and our analysis will be broken down by disease, researcher academic discipline, impact and years of experience. \\


\noindent Our principal objective is to measure the perceived likelihood/feasibility and time-frame of eradication of certain neglected tropical diseases and malaria among those who are professional researchers of those respective diseases, at a larger scale than any previous study. Our secondary objective is to examine the relationship between the perceived likelihood/feasibility of eradication of diseases with the respective attention allotted to them in both the popular and academic literature. Our tertiary objective is to establish which specific areas of knowledge are lacking through an examination of researcher characteristics (academic discipline, geography, etc.) insofar as those characteristics are associated with differential perceptions regarding time-to-eradication.


\subsection*{Methods and Design}
\addcontentsline{toc}{subsection}{Methods and Design}

We will “webscrape” from PubMed the authors, abstracts, and journal information of all articles related to disease X using standardized search terms. We will then send emails to all first, last, and corresponding authors (whose addresses can be located), asking 2 simple questions:  
\begin{enumerate}
\item In your opinion, how many years will it be until disease X is eradicated? (0-99+)
\item Please rank the following ten areas in order of where attention is most needed in order to achieve eradication (10 = attention most needed; 1 = attention least needed).
\end{enumerate}

\noindent These questions can also be answered via an online survey: \href{http://goo.gl/forms/Ib80IwgwQY}{http://goo.gl/forms/Ib80IwgwQY}. \\


\noindent We will then compile a database which links researcher meta-information (percent and number of publications in top-decile journals, publication quantity, geography of institution, geography of research focus, gender, academic discipline) with their surveyed attitudes regarding eradication (years-to-eradication and ordered ranking of factors). \\


\noindent The design of this study is typical, but this study is noteworthy in two areas: (1) its scale (by using automated web-scraping, emailing, and surveying, we will reach the maximum number of experts), and (2) its democratic approach (we assume that the more experts’ opinions reflected, the closer we are to approximating the “truth”). Our results will be of value not only to the scientific community, but also to policy-makers and public health practitioners. By gauging and synthesizing the “wisdom of (informed) crowds”, we aim to establish a barometer of scientific opinion in a manner that is fully reproducible.


\subsection*{Evaluation criteria}
\addcontentsline{toc}{subsection}{Evaluation criteria}


\noindent \textbf{1. What are the ethical considerations that need to be addressed and how will they be addressed?} \\
We will not be collecting personal health information, or any biological samples. Nor will we be dealing in any way, shape or form with health outcomes or treatment data.
We will only contact researchers whose information is publicly available online.   \\

\noindent The only potential area of “sensitive” information pertains to the disclosure of researchers’ opinions. However, we will state clearly in both the “invitation to participate” email as well as in the online survey form that results will be made fully public; researchers who choose not to participate are free to do so, and will not be contacted thereafter.  \\


\noindent \textbf{2. List the ethics committees (both human and/or animal) which either have reviewed or will review this proposal.} \\


\noindent None. Given the nature of this study, no human/animal ethics committees’ review is necessary. \\

\noindent \textbf{3. Describe the expertise required for the project and which member(s) of the research team will provide each area of expertise.} \\


\noindent Area expertise in malaria: Elisa Sicuri and Joe Brew \\
\noindent Area expertise in NTDs: Elisa Sicuri and Celine Aerts \\


\noindent \textbf{4. How does the proposal fit in with ISGlobal’s scientific agenda?} \\ 


\noindent ISGlobal is a thought leader in the areas of both malaria and NTDs (research programmes in Chagas, and, increasingly, leishmaniasis) as well as their corresponding eradication movements. By using modern, technologically-oriented means to establish a “barometer” of international researcher consensus on the perceived feasibility and time-frame of eradication, ISGlobal would cement its position at the center of the ongoing international dialogue on the subject. Furthermore, given ISGlobal’s stake in eradication campaigns, the results of this study could (a) inform which disciplines and diseases have the most “research gaps” to be filled in order to achieve eradication, (b) identify areas of consensus and discord between different disciplines, (c) provide (crowd-informed) estimates of the timeline to eradication. \\

\noindent \textbf{Budget estimation and expected source of funding for this study.} \\

\noindent None. Given that this topic is directly relevant to the PhD-specific research of the two co-Pis (Celine Aerts and Joe Brew), no project-specific funding is required. \\

\noindent \textbf{Other comments} \\

\noindent Have all co-investigators read and approved this proposal? YES \\
\noindent Do you expect to handle samples of human origin in the study? NO \\
\noindent Do you expect to handle personal information in the study? NO. 




\newpage


\bibliography{library}{}
\bibliographystyle{apalike}  

\end{document}
